\section{Introduction}

In various kinds of diseases, it is important to choose the correct treatment at
the current time point. Since the human body is too complex to be described
directly using mathematics, special features are modeled with mathematical
systems such as the Mackey-Glass equations for respiratory and hematopoietic
diseases \cite{mackey1977}. These models make use of available data about
previous information in order to reduce the uncertainty under which a decision
is made. This is especially important when the consequences of a decision are
severe, for example the decision of breast cancer treatment after radiation.
This particular problem can be addressed by using a stochastic differential
equation, as done by Oroji \textit{et al} \cite{oroji2016}.

In different kinds of biological models, even chaotic behavior is observed. The
Mackey-Glass equations possess chaotic behavior for certain choices of their
parameters, as shown by Farmer \cite{farmer1982}. That means that even small
measurement errors of the initial state lead to exponentially increasing
errors of the time series prediction. Because of that, it is important
that the time series prediction should be as robust as possible against noise.

This gets important in particular because biological data cannot be measured 
without the presence of noise. In
this report, we show that different kinds of noise assumptions can impact the
results of time series prediction. We evaluate all our investigated problems
with regard to the popular independent and identically distributed ($i.i.d.$)
random noise as well as noise from a
discretized Wiener process.
Since mathematical models of biological systems cannot describe their
characteristics perfect and error-free, it is assumed that the latter type of
noise is more realistic. But unfortunately this kind of noise increases the
difficulty of time series prediction, even for simple mathematical functions,
as seen in Section~\ref{sec:sine}.

The remaining report is structured as follows. First, we perform proper time
series prediction on the sinus function and show how different noise models
impact the prediction ability of our network models. After that, we show how
different levels of chaos impact the possibility to forecast the development
of time series at the example of the Mackey-Glass time series. We show
how the variance and the type
of noise impacts the possibility to predict deterministic
chaotic time series at the example of this time series. Last
but not least, we report fundamental limitations of neural networks to
approximate stochastic time series at the example of biological oscillators.
