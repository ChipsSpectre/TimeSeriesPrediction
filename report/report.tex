\documentclass{article}

\title{Challenges in stochastic time series prediction}
\author{Maximilian Hornung, Jens Settelmeier}

\begin{document}
\maketitle

\begin{abstract}
    Predicting the future development of time series is of interest in different
    areas of computational biology. The time series in biological models exhibit
    challenging characteristics such as chaotic or stochastic behavior. In this
    report, time series prediction is done on different kind of biological time
    series using deep neural networks. In our evaluation, we identify challenges
    and limitations of this approach, and compare different architectures of
    deep neural networks with regard to their performance.
\end{abstract}

\section{Introduction}

The remaining report is structured as follows. First, we perform proper time 
series prediction on the $sine$ function and show how different noise models
impact the prediction ability of our network models. We conduct an evaluation
how these results can be extended to the prediction of 
ordinary differential equations (\textbf{ODE}s) at the example of the 
differential equations of the harmonic oscillator. After that, we show how the
stochasticity of noise impactss the possibility to predict deterministic
chaotic time series at the example of the \emph{Mackey Glass} time series. Last
but not least, we report how the poor results of numerical approximations on 
stocastic differential equations (\textbf{SDE}s) can be explained based on our
previous results.

\section{Methods}

\section{Evaluation}
\subsection{Time series prediction of continuous functions}

\subsection{Time series prediction of ODEs}
 
\subsection{Mackey Glass time series prediction}

In order to model diseases related to dynamic respiratory and hematopoietic 
diseases, Mackey \textit{et al.} proposed the mackey-glass equations, a kind of 
first-order nonlinear differential delay equations \cite{mackey1977}. If the 
delayed time ($x_{\tau} = x(t - \tau)$) exceeds the delay $\tau > 16.8$, then 
equation \ref{equ:mackey} behaves chaotic \cite{farmer1982}.

\begin{equation}
  \frac{dx}{dt} = \beta \cdot \frac{x_{\tau}}{1 + x_{\tau}^n}
  \label{equ:mackey}
\end{equation}

The first approach to predict the short-time behavior of chaotic time series
was done by Farmer \textit{et al.} who proposed a \texttt{local approximation}
technique \cite{farmer1987}. After improval of predictions using support vector
machines by Müller \textit{et al.} \cite{muller1997}, the focus in research
shifted towards artifical neural networks which enable even better predictions.
Two of the latest developments are the usage of Wavelet Networks
\cite{alexandridis2013} and particle swarm optimization \cite{caraballo2016}.


\subsection{SDE time series prediction}
    
\section{Conclusion}

\bibliographystyle{plain}
\bibliography{report}
\end{document} 